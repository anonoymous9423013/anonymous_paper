\begin{tabular}{lcccccccc}
\toprule
         Method &  Simulations &   F1 p-value &   F2 p-value &   F3 p-value &   F4 p-value &  F1 A^12 &  F2 A^12 &  F3 A^12 &  F4 A^12 \\
\midrule
Baseline-Manual &         4379 & 1.776357e-15 & 1.776357e-15 & 1.776357e-15 & 1.776357e-15 &   0.0492 &      0.0 &      0.0 &   0.3168 \\
    fastFitness &         7000 & 1.776357e-15 & 1.776357e-15 & 1.776357e-15 & 1.776357e-15 &   0.0124 &      0.0 &      0.0 &   0.2116 \\
\bottomrule
\end{tabular}
